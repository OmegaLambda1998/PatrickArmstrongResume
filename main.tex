%-------------------------
% Resume in Latex
% Author : Jake Gutierrez
% Based off of: https://github.com/sb2nov/resume
% License : MIT
%------------------------

\documentclass[letterpaper,11pt]{article}

\usepackage{latexsym}
\usepackage[empty]{fullpage}
\usepackage{titlesec}
\usepackage{marvosym}
\usepackage[usenames,dvipsnames]{color}
\usepackage{verbatim}
\usepackage{enumitem}
\usepackage[hidelinks]{hyperref}
\usepackage{fancyhdr}
\usepackage[english]{babel}
\usepackage{tabularx}
\input{glyphtounicode}
\usepackage{fontawesome5}
\usepackage{etoolbox}
\usepackage{xifthen}
\usepackage{makecell}


%----------FONT OPTIONS----------
% sans-serif
% \usepackage[sfdefault]{FiraSans}
% \usepackage[sfdefault]{roboto}
% \usepackage[sfdefault]{noto-sans}
% \usepackage[default]{sourcesanspro}

% serif
% \usepackage{CormorantGaramond}
% \usepackage{charter}


\pagestyle{fancy}
\fancyhf{} % clear all header and footer fields
\fancyfoot{}
\renewcommand{\headrulewidth}{0pt}
\renewcommand{\footrulewidth}{0pt}

% Adjust margins
\addtolength{\oddsidemargin}{-0.5in}
\addtolength{\evensidemargin}{-0.5in}
\addtolength{\textwidth}{1in}
\addtolength{\topmargin}{-.8in}
\addtolength{\textheight}{1.2in}

\urlstyle{same}

\raggedbottom
\raggedright
\setlength{\tabcolsep}{0in}

% Sections formatting
\titleformat{\section}{
  \vspace{-4pt}\scshape\raggedright\large
}{}{0em}{}[\color{black}\titlerule \vspace{-5pt}]

\titleformat{\subsection}{
  \vspace{-4pt}\scshape\raggedright\normalsize
}{}{0em}{}[\color{black}\titlerule \vspace{-5pt}]
% Ensure that generate pdf is machine readable/ATS parsable
\pdfgentounicode=1

%-------------------------
% Custom commands

\newcommand*\lbreak{\\[\baselineskip]}

\newcommand{\resumeItem}[1]{
  \item\small{
    {#1 \vspace{-2pt}}
  }
}

\newcommand{\resumeSubheading}[4]{
  \vspace{-2pt}\item
    \begin{tabular*}{0.97\textwidth}[t]{l@{\extracolsep{\fill}}r}
      \textbf{#1} & #2 \\
      \textit{\small#3} & \textit{\small #4} \\
    \end{tabular*}\vspace{-7pt}
}

\newcommand{\resumeSubSubheading}[2]{
    \item
    \begin{tabular*}{0.97\textwidth}{l@{\extracolsep{\fill}}r}
      \textit{\small#1} & \textit{\small #2} \\
    \end{tabular*}\vspace{-7pt}
}

\newcommand{\resumeProjectHeading}[2]{
    \item
    \begin{tabular*}{0.97\textwidth}{l@{\extracolsep{\fill}}r}
      \small#1 & #2 \\
    \end{tabular*}\vspace{-7pt}
}

\newcommand{\resumeSubItem}[1]{\resumeItem{#1}\vspace{-4pt}}

\renewcommand\labelitemii{$\vcenter{\hbox{\tiny$\bullet$}}$}

\newcommand{\resumeSubHeadingListStart}{\begin{itemize}[leftmargin=0.15in, label={}]}
\newcommand{\resumeSubHeadingListEnd}{\end{itemize}}
\newcommand{\resumeItemListStart}{\begin{itemize}}
\newcommand{\resumeItemListEnd}{\end{itemize}\vspace{-5pt}}

\newcommand{\insertName}{Patrick Armstrong}
\newcommand{\insertWebsite}{\href{https://www.omegalambda.au/}{\faIcon{globe}~\underline{omegalambda.au}~}}
\newcommand{\insertEmail}{\href{mailto:patrick.armstrong@anu.edu.au}{\faIcon{envelope}~\underline{patrick.armstrong@anu.edu.au}~}}
\newcommand{\insertGithub}{\href{https://www.github.com/OmegaLambda1998}{\faIcon{github}~\underline{OmegaLambda1998}~}}
\newcommand{\insertOrcid}{\href{https://www.orcid.org/000-003-1997-3649}{\faIcon{orcid}~\underline{0000-0003-1997-3649}}}

\newcommand{\educationElement}[4]{%
    \resumeSubHeadingListStart
        \resumeSubheading
            {#1}
            {#2}
            {#3}
            {#4}
    \resumeSubHeadingListEnd
}

\newcommand{\experienceElement}[5]{%
    \resumeSubHeadingListStart
        \resumeSubheading
            {#1}
            {#2}
            {\makecell[l]{#3}}
            {#4}
            {#5}
            % \ifthenelse{\isempty{#5}}{}{%
                % \resumeItemListStart
                    % \renewcommand*{\do}[1]{\resumeItem{##1}}
                    % \docsvlist{#5}%
                % \resumeItemListEnd
            % }%
    \resumeSubHeadingListEnd
}

\newcommand{\awardElement}[5]{%
    \resumeSubHeadingListStart
        \resumeSubheading
            {#1}
            {#2}
            {#3}
            {#4}
            \ifthenelse{\isempty{#5}}{}{%
                \resumeItemListStart
                    \renewcommand*{\do}[1]{\resumeItem{##1}}
                    \docsvlist{#5}%
                \resumeItemListEnd
            }%
    \resumeSubHeadingListEnd
}

\newcommand{\conferenceElement}[5]{%
    \resumeSubHeadingListStart
        \resumeSubheading
            {#1}
            {#2}
            {#3}
            {#4}
            \ifthenelse{\isempty{#5}}{}{%
                \resumeItemListStart
                    \renewcommand*{\do}[1]{\resumeItem{##1}}
                    \docsvlist{#5}%
                \resumeItemListEnd
            }%
    \resumeSubHeadingListEnd
}

\newcommand{\communicationElement}[5]{%
    \resumeSubHeadingListStart
        \resumeSubheading
            {\makecell[l]{#1}}
            {\makecell[r]{#2}}
            {\makecell[l]{#3}}
            {\makecell[r]{#4}}
            \ifthenelse{\isempty{#5}}{}{%
                \resumeItemListStart
                    \renewcommand*{\do}[1]{\resumeItem{##1}}
                    \docsvlist{#5}%
                \resumeItemListEnd
            }%
    \resumeSubHeadingListEnd
}

\newcommand{\proficiencyElement}[5]{%
    \resumeSubHeadingListStart
        \resumeSubheading
            {#1}
            {#2}
            {\makecell[l]{#3}}
            {#4}
            {#5}
            % \ifthenelse{\isempty{#5}}{}{%
                % \resumeItemListStart
                    % \renewcommand*{\do}[1]{\resumeItem{##1}}
                    % \docsvlist{#5}%
                % \resumeItemListEnd
            % }%
    \resumeSubHeadingListEnd
}

\newcommand{\publicationElement}[5]{%
    \textbf{{#1}} ({#4})\\{#3}; {#2}\lbreak{}
}

%\newcommand{\publicationElement}[5]{%
%    \resumeSubHeadingListStart
%        \resumeSubheading
%            {\makecell[l]{#1}}
%            {\makecell[r]{#2}}
%            {\makecell[l]{#3}}
%            {\makecell[rt]{#4}}
%            \ifthenelse{\isempty{#5}}{}{%
%                \resumeItemListStart
%                    \renewcommand*{\do}[1]{\resumeItem{##1}}
%                    \docsvlist{#5}%
%                \resumeItemListEnd
%            }%
%    \resumeSubHeadingListEnd
%}

%-------------------------------------------
%%%%%%  RESUME STARTS HERE  %%%%%%%%%%%%%%%%%%%%%%%%%%%%


\begin{document}

\begin{center}
    \textbf{\Huge \scshape \insertName} \\ \vspace{2pt}
    \insertWebsite $|$ \insertEmail $|$ 
    \insertGithub $|$
    \insertOrcid
\end{center}


%-----------EDUCATION-----------
\section{Education}
  \educationElement{Doctor of Philosophy (Astronomy \& Astrophysics)}{Australian National University}{}{February 2020 -- Present}
  \educationElement{Bachelor of Science (Adv.) (Hon.)}{Australian National University}{}{February 2016 -- October 2019}


%-----------EXPERIENCE-----------
\section{Academic Experience}

    \experienceElement{DES Builder}{Dark Energy Survey}{Develop \& maintain the~\href{https://github.com/dessn/Pippin}{\underline{Pippin}} pipeline, Internal review of DES papers, Organise \& host meetings.}{July 2023 -- Present}{}

    \experienceElement{MSATT Student Mentor}{MSATT}{Provide guidance and mentorship for highschool students completing astronomical projects.}{February 2020 -- Present}{}

    \experienceElement{Astronomical Tutor}{Australian National University}{Sole tutor for~\textit{Galaxies and Cosmology (ASTR3002)}.}{July 2019 -- October 2022}{}

\section{Other Experience}

    \experienceElement{Student Seminar Planning Committee Member}{Australian National University}{[2022]: Senior planning committee member, [2021]: Planning committee member}{February 2021 -- October 2022}{}

    \experienceElement{Mt. Stromlo Outreach Officer}{Australian National University}{Deliver high quality outreach experience for school groups and families}{January 2018 -- December 2022}{}

    \experienceElement{Astronomical Consultant}{}{[2020] Research for Questacon's Australia in Space exhibition\\{}[2019] Research for Penguin Random House's Stargazer publication\\{}[2018] Research and preparation for ABC's Stargazing Live 2018\\{}[2018] Building backend code and moderation for the Skymapper Citizen Science Project: Supernova Sighting}{}{}
    \experienceElement{Questacon Staff}{Questacon}{[2019 -- 2020] Learning Programs Presenter (APS 4)\\{}[2016 -- 2019] Questacon Assistant (APS 2)\\{}[2015 -- 2019] Gallery Assistant (APS 1)}{}{}


\section{Recognition \& Distinctions}

    \awardElement{ANU 2.3m Observing Time}{Siding Spring Observatory}{The Ultimate Low-z Supernova Sample for Cosmology}{2023}{}

    \awardElement{Alex Rodgers Travelling Scholarship}{ANU College of Science}{Travel to DES Collaboration Meeting 2022}{2022}{}
    
    \awardElement{Commendation for Excellence in Tutoring or Demonstrating}{ANU College of Science}{Tutoring~\textit{Galaxies and Cosmology (ASTR3002)}}{2022}{}

    \awardElement{NCI ANU Merit Allocation Scheme}{GADI}{Forward Modelling Supernova Cosmology}{2021 -- 2022}{}
    
    \awardElement{Australian Government Research Training Program}{Australian National University}{PhD Scholarship}{2020 -- Present}{}
    
    \awardElement{RSAA Supplementary Scholarship}{Australian National University}{PhD Scholarship}{2020 -- Present}{}
    
    \awardElement{ANU Science, Health, and Medicine Honours Scholarship}{Australian National University}{Honours Scholarship}{2019}{}
    
    \awardElement{ANU Summer Research Scholarship}{Australian National University}{Develop a TNS Bulk Report API for the SkyMapper Transient Survey}{2016}{}
    
    \awardElement{Boyapti Computer Science and Mathematics prize for first year}{Australian National University}{Top grades in mathematics and computing}{2016}{}

\section{Technical Skills}

    \proficiencyElement{Python 3}{Numpy, Scipy, Pandas, Emcee, GetDist, Ultranest}{\href{https://github.com/OmegaLambda1998/Covariance}{\underline{Covariance Matrix Calculator}}, \href{https://github.com/dessn/Pippin}{\underline{Pippin Pipeline}}, \href{https://github.com/dessn/BiasValidation}{\underline{Cosmology Validator}}}{}{}
    \proficiencyElement{Julia}{Makie, PyCall, Unitful, AffineInvariantMCMC}{\href{https://github.com/OmegaLambda1998/ShockCooling.jl}{\underline{Fit Supernovae Lightcurve}}, \href{https://github.com/OmegaLambda1998/SALTJacobian.jl}{\underline{Approximate Supernovae Simulations}}}{}{}
    \proficiencyElement{HTML, CSS, \& Javascript}{Django, Franklin}{\href{https://www.mso.anu.edu.au/debass/}{\underline{DEBass Survey}}, \href{https://www.omegalambda.au/}{\underline{Personal Website}}}{}{}
    \proficiencyElement{Statistics}{}{MCMC, ABC, and other Bayesian Inference, Frequentist Inference, Data Analysis, Data Visualisation}{}{}


\section{Publications}

    \subsection{First Author}

    \subsection{Co-Author}

\newpage{}\section{Communcation}

    \communicationElement{SN2017jgh: a high-cadence complete\\shock cooling light curve of a SN IIb\\with the Kepler telescope}{Over 180 items in print, radio, and online,\\across Australia and internationally}{\textbf{Highlights:}, Al Jazeera, National Geographic Indonesia, Radio Canada,\\De Morgen, ABC Science online, The Guardian, Space Australia,\\Sky News Australia, 2GB and on the AAP wires}{2021}{}

\section{Conference Talks}

    \conferenceElement{CosmoPalooza}{Invited Speaker}{DES SN 5 Year Methodology \& Results}{2023}{}
    \conferenceElement{DES Collaboration Meeting}{Invited Speaker}{DES 5 year supernova analysis}{2020, 2021, 2022, 2023}{}
    \conferenceElement{ASA Annual Science Meeting}{Speaker}{DES 5 year supernova analysis}{2020, 2021, 2022, 2023}{}
    \conferenceElement{Kepler K2 Extragalactic Data Analysis Meeting}{Attendee}{Investigating transients in Kepler's K2 survey}{2018}{}

%-------------------------------------------
\end{document}
